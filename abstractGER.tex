\chapter*{Einleitung}
Seit einigen Jahren unterliegt die Softwareentwicklung in vielen Unternehmen einem Wandel. Wurden bisher Anwendungen "`aus einem Guss"' entwickelt, entfernt sich die IT-Branche mehr und mehr von diesem monolithischen Gedanken und wendet sich einer modularen Anwendungsentwicklung zu. Anwendungen bestehen aus Microservices. Kleine Module einer Anwendung, die unabhängig voneinander bestimmte Funktionalitäten der Anwendung abdecken und im Zusammenspiel die Anwendung ausmachen. Diese Module können wiederum aufgrund ihrer Unabhängigkeit erweitert und angepasst werden, ohne dass die Funktionalität der Gesamtanwendung beeinträchtigt wird. Ein großer Vorteil gegenüber der bisherigen monolithischen Anwendung, welche für Anpassungen kurzzeitig komplett nicht zur Verfügung steht. Bei umfangreicheren Anwendungen kann die modulare Aufbauweise jedoch auch zu einer erhöhten Komplexität führen. Das Prinzip der Microservices macht sich Amazon Web Services ebenfalls zunutze. Web Services bieten eine schwach gekoppelte, asynchron und nachrichtenbasierte Kommunikation und bilden damit die Grundlage für ein weltweit verfügares Netz heterogener Ressourcen. \cite{computerwoche:reder}, \cite{baun:cloudcomp} \\
Eine weitere Bewegung ist in der Zusammenarbeit von Softwareentwicklern und Systemadministratoren zu beobachten. Letztere stellen IT-Ressourcen bereit, damit die Anwendungen der Entwickler mit guter Performance laufen. Absprachen untereinander und notwendige Systemanpassungen führten zu längeren Releasezyklen einer Anwendung (z.B. monatlich). Durch dieses Vorgehen war es Entwicklern nicht möglich auf Anpassungswünsche der Kunden zeitnah zu reagieren. "`DevOps"' änderte die obige Herangehensweise und brachte Entwickler und Administratoren näher zusammen. Im Fokus steht die Automatisierung von Tests und Bereitstellungsprozessen für Anwendungen und IT-Ressourcen. Es sind mehrere Deployments pro Tag möglich, wenn die Bereitstellung der Anwendung bestmöglich automatisch abläuft. Vom Erstellen der Anwendung über automatische Tests bis zur automatischen Installation in der Testumgebung bzw. schlussendlich dem Produktivsystem. \\
Infrastructure as Code ermöglicht die codebasierte Konfiguration benötigter Ressourcen sowie die Möglichkeit der Versionierung der Konfigurations-Code-Teile. Voraussetzung ist Hardware, die mit diesem Code auch umgehen kann. Amazon Web Services bietet eine solche Hardware und eine entsprechende Schnittstelle, die diesen Code entgegennehmen und zur Verarbeitung geben kann. Somit sind bei Bedarf in wenigen Minuten neue Ressourcen verfügbar. \cite{wittig:awsinaction} \\
Durch dieses Vorgehen kann ein Unternehmen zeitnah auf Anforderungen reagieren, was sicherlich dazu beitragen kann, um sich am Markt zu behaupten.
