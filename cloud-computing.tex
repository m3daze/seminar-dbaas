\chapter{Cloud Computing}\label{chapter:kapitellabel} %%%%%%%%%%%%%%%%%%%%%%%%%%%%
Der Begriff Cloud Computing besitzt keine standardisierte Definition, weshalb
er vielseitig interpretierbar ist. Eine oft zitierte Definition stammt vom
Nationalen Institut für Standards und Technologie (NIST) \cite{baun:cloudcomp} und beschreibt
\begin{itemize}
  \item fünf wesentliche Eigenschaften
  \item drei verschiedene Dienstklassen
  \item vier unterschiedliche Betriebsmodelle
\end{itemize}
die für alle Cloud Computing Angebote gelten.
Die fünf Eigenschaften sind:
\begin{enumerate}
  \item Diensterbringung auf Anforderung
  \item Netzwerkbasierter Zugang
  \item Ressourcen-Pooling
  \item Elastizität
  \item Messbare Dienstqualität
\end{enumerate}

Der Begriff "`Cloud"' ist als Metapher zu verstehen, welche beschreibt, dass diverse Anbieter über das Internet (oder Intranet eines Unternehmens) ihre Dienste zur Verfügung stellen. Trotz unterschiedlicher
Interpretationsmöglichkeiten gibt es grundlegende Ziele des Cloud Computing,
die alle einen.
\begin{enumerate}
  \item Cloud Computing beschreibt die dynamische Bereitstellung und Nutzung von IT-Ressourcen,
  Plattformen und Anwendungen als elektronisch verfügbare Dienste, unter der Nutzung
  von Virtualisierung und dem modernen Web.
  \item Die bereitgestellten Dienste sollen durch mehrere Nutzer skalierbar verwendbar
  sein. Das bedeutet, sie sind sowohl auf Abruf als auch nach Bedarf verfügbar.
\end{enumerate}
\cite{baun:cloudcomp}, \cite{wittig:awsinaction}
\\
\\
Die IT-Ressourcen selbst sind für den Nutzer nicht direkt ersichtlich. Das
Abstraktionslevel der Cloud variiert von virtueller Hardware bis hin zu komplexen verteilten Systemen.
\\ Die Nutzung der Cloud-Dienste bietet einige Vorteile für den Anwender. Ein Vorteil
ist die dynamische Skalierbarkeit der Dienste, weshalb
sie von jungen Startups bis hin zu großen Unternehmen genutzt werden. Die sorgfältige Planung an zukünftig notwendigen IT-Ressourcen weicht dem On-Demand Ansatz. Es werden nur so viele Ressourcen bereitgestellt, wie auch benötigt werden. Abgerechnet wird nach dem 'pay-per-use'--Prinzip, welches ein weiterer Vorteil ist.
Nach diesem Prinzip werden nur die tatsächlich genutzten Ressourcen abgerechnet.
Nicht mehr und nicht weniger. Daneben ist es ein großer Vorteil, dass die IT-Ressourcen
selbst in der Regel virtualisiert sind. Damit gibt es keine zu beachtenden systembedingten
Abhängigkeiten. Ebenso entfallen mögliche Zwangsbedingungen für die Anwendungen des Nutzers
\cite{baun:cloudcomp}, \cite{wittig:awsinaction}.
\\ Cloud Computing bietet auf lange Sicht die Perspektive, das klassische Rechenzentrum zu einem
IT-Servicezentrum umzuwandeln. Durch die immer spezialisierteren Dienste werden
Mitarbeiter aus dem Management befähigt, eigenständig benötigte IT-Ressourcen zu kaufen.
Dabei kann die mitunter aufwendige Abstimmung mit der internen IT-Abteilung deutlich
geringer oder ganz ausfallen,
was zu einer Veränderung in der Rolle der IT aber auch des Managements führen kann \cite{baun:cloudcomp}.

% keep an blank line above
